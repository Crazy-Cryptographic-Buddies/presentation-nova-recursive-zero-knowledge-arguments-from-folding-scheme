% !TeX root = main.tex
\begin{frame}{R1CS and Folding Attempt}
	$\mathbf{A}, \mathbf{B}, \mathbf{C}$: matrices over $\mathbb{F}$.
	
	An R1CS instance is a vector $\mathbf{z}$ satisfying $\mathbf{A}\mathbf{z}\circ\mathbf{B}\mathbf{z} = \mathbf{C}\mathbf{z}$.
	
	Assume we have $\mathbf{z}_1, \mathbf{z}_2$ satisfying $$\mathbf{A}\mathbf{z}_1\circ\mathbf{B}\mathbf{z}_1 = \mathbf{C}\mathbf{z}_1 \text{ and } \mathbf{A}\mathbf{z}_2\circ\mathbf{B}\mathbf{z}_2 = \mathbf{C}\mathbf{z}_2.$$
	
	We want to obtain folded instance $\mathbf{z} = \mathbf{z}_1 + r\cdot\mathbf{z}_2$ for some random challenge $r \in \mathbb{F}$.
	
	However, $\mathbf{A}\mathbf{z}\circ \mathbf{B}\mathbf{z} \not= \mathbf{C}\mathbf{z}$.
\end{frame}

\begin{frame}{Relaxed R1CS and Folding}
	$\mathbf{A}, \mathbf{B}, \mathbf{C}$: matrices over $\mathbb{F}$.
	
	A relaxed R1CS instance $(u, \mathbf{z}, \mathbf{e})$ satisfies $\mathbf{A}\mathbf{z}\cdot \mathbf{B}\mathbf{z} = u\mathbf{C}\mathbf{z} + \mathbf{e}$. 
	
	Fold $2$ instances $(u_1, \mathbf{z}_1, \mathbf{e}_1)$ and $(u_2, \mathbf{z}_2, \mathbf{e}_2)$ by observing
	\begin{equation*}
		\begin{aligned}
			\mathbf{A}\explain{(\mathbf{z}_1 + r\cdot \mathbf{z}_2)}{\mathbf{z}} \circ\mathbf{B}\explain{(\mathbf{z}_1 + r\cdot \mathbf{z}_2)}{\mathbf{z}} = \explain{(u_1 + r\cdot u_2)}{u}\cdot\mathbf{C}\explain{(\mathbf{z}_1 + r\cdot \mathbf{z}_2)}{\mathbf{z}} + \mathbf{e}
		\end{aligned}
	\end{equation*}
	where $\mathbf{e} = \mathbf{e}_1 + r\cdot \explain{(\mathbf{A}\mathbf{z}_1\circ\mathbf{B}\mathbf{z}_2 + \mathbf{A}\mathbf{z}_2 \circ\mathbf{B}\mathbf{z}_1 - u_1\cdot\mathbf{C}\mathbf{z}_2 - u_2\cdot\mathbf{C}\mathbf{z}_1)}{\mathbf{t}} + r^2\cdot\mathbf{e}_2$.
	
	Hence, $\mathbf{A}\mathbf{z}\circ\mathbf{B}\mathbf{z} = u\cdot \mathbf{C}\mathbf{z} + \mathbf{e}$ and $\mathbf{e} = \mathbf{e}_1 + r\cdot\mathbf{t} + r^2\cdot\mathbf{e}_2$.
\end{frame}

\begin{frame}{Folding Scheme for Relaxed R1CS}
	Use additively homomorphic commitment scheme. Commitment to $\mathbf{x}$ is denoted by $\overline{\mathbf{x}}$.
	
	\textbf{Common inputs:} $(u_1,\overline{\mathbf{z}_1},\overline{\mathbf{e}_1}),(u_2, \overline{\mathbf{z}_2}, \overline{\mathbf{e}_2})$. 
	
	\textbf{$\Prover$'s inputs:} $\mathbf{z}_1, \mathbf{z}_2, \mathbf{e}_1, \mathbf{e}_2$ and corresponding randomness.
	
	\textbf{Purpose:} obtain folded instance $\mathbf{z}$.
	
	\begin{itemize}
		\item $\Prover$ sends $\overline{\mathbf{t}}$ to $\Verifier$.
		\item $\Verifier$ sends a challenge $r \in \mathbb{F}$ to $\Prover$.
		\item Both $\Prover, \Verifier$ compute $u := u_1 + r\cdot u_2$, $\overline{\mathbf{z}} := \overline{\mathbf{z}_1} + r\cdot \overline{\mathbf{z}_2}$ and $\overline{\mathbf{e}} := \overline{\mathbf{e}_1} + r\cdot\overline{\mathbf{t}} + r^2\cdot \overline{\mathbf{e}_2}$.
		\item $\Prover$ outputs $u, \mathbf{z}$ and $\mathbf{e}$.
	\end{itemize}  

	\textbf{Non-Interactivity.} Use random oracle to sample $r$.
\end{frame}